% Definiciones y constantes de estilo
% Clase del documento
\documentclass[a4paper,12pt,twoside,openright,titlepage]{book}

%
% Paquetes necesarios
%

% Símbolo del euro
\usepackage{eurosym}
% Codificación UTF8
\usepackage[utf8]{inputenc}
% Caracteres del español
\usepackage[spanish]{babel}
% Código, algoritmos, etc.
\usepackage{listings}
% Definición de colores
\usepackage{color}
% Extensión del paquete color
\usepackage[table,xcdraw]{xcolor}
% Márgenes
\usepackage{anysize}
% Cabecera y pie de página
\usepackage{fancyhdr}
% Estilo título capítulos
\usepackage{quotchap}
% Algoritmos (expresarlos mejor)
\usepackage{algorithmic}
% Títulos de secciones
\usepackage{titlesec}
% Fórmulas matemáticas
\usepackage[cmex10]{amsmath}
% Enumeraciones
\usepackage{enumerate}
% Páginas en blanco
\usepackage{emptypage}
% Separación entre cajas
\usepackage{float}
% Imágenes
\usepackage[pdftex]{graphicx}
% Mejora de las tablas
\usepackage{array}
% Mejora de los símbolos matemáticos
\usepackage{mdwmath}
% Separar figuras en subfiguras
\usepackage[caption=false,font=footnotesize]{subfig}
% Incluir pdfs externos
\usepackage{pdfpages}
% Mejoras sobre las cajas
\usepackage{fancybox}
% Apéndices
\usepackage{appendix}
% Marcadores (para el pdf)
\usepackage{bookmark}
% Estilo de enumeraciones
\usepackage{enumitem}
% Espacio entre líneas y párrafos
\usepackage{setspace}
% Glosario/Acrónimos
\usepackage[acronym]{glossaries}
% Fuentes
\usepackage[T1]{fontenc}
% Bibliografía
\usepackage[sorting=none,natbib=true,backend=bibtex,bibencoding=ascii]{biblatex}
% Fix biblatex+babel warning
\usepackage{csquotes}
% enlaces
\usepackage{hyperref}
\usepackage{multirow}
\usepackage{afterpage}


%%Cuestionario 

\newdimen\scalewidth
\scalewidth=0.3\hsize

\def\questioncheks#1\par#2\par{\hbox to \hsize
  {\vbox{\hsize=0.45\hsize #1\dotfill}\quad#2\hfil}\medskip\goodbreak}

\def\question#1\par#2\par{\hbox to \hsize
  {\vbox{\hsize=0.77\hsize #1\dotfill}\quad#2\hfil}\medskip\goodbreak}

\def\fiveboxes#1#2#3#4#5{\hbox to\scalewidth
    {\boxit{#1}\hfil\boxit{#2}\hfil\boxit{#3}\hfil%
     \boxit{#4}\hfil\boxit{#5}}}

\def\boxit#1{\hbox{\lower0.7ex\vbox{\hrule\hbox{\vrule\kern1pt
    \vbox{\kern1pt\hbox to 1.4em
    {\small\strut\hfil #1\hfil}\kern1pt}\kern1pt\vrule}\hrule}}}

\def\xagree{\xscale{totalmente en desacuerdo}{totalmente de acuerdo}}
\def\boxes{\fiveboxes{1}{2}{3}{4}{5}\ignorespaces}
\def\xscale#1#2{%
    \setbox0=\hbox{\boxes}%
    \setbox1=\hbox to \wd0{\small\strut\hfill #2 $\to$}%
    \setbox2=\hbox to \wd0{\small\strut $\gets$ #1 \hfill}%
    \vbox{\vbox to 0pt{\vss\box1\box2\kern2pt}\vbox{\box0}}}

\def\freequestion#1\par{#1\par\nobreak
    \begingroup\nobreak
    \advance\leftskip by 3pc
    \hrule width 0pt height 1.7\baselineskip\hrulefill
    \hrule width 0pt height 1.7\baselineskip\hrulefill
    \hrule width 0pt height 1.7\baselineskip\hrulefill
    \par
    \medskip
    \endgroup
}

\def\optquestion#1\par{#1\par\nobreak
    \begingroup\nobreak
    \advance\leftskip by 3pc
    \hrule width 0pt height 1.7\baselineskip\hrulefill
    \par
    \medskip
    \endgroup
}

% Enlaces
\hypersetup{hidelinks,pageanchor=true,colorlinks,citecolor=Fuchsia,urlcolor=black,linkcolor=Cerulean}

% Euro (€)
\DeclareUnicodeCharacter{20AC}{\euro}

% Inclusión de gráficos
\graphicspath{{./graphics/}}

% Texto referencias
\addto{\captionsspanish}{\renewcommand{\bibname}{Bibliografía}}

% Extensiones de gráficos
\DeclareGraphicsExtensions{.pdf,.jpeg,.jpg,.png}

% Definiciones de colores (para hidelinks)
\definecolor{LightCyan}{rgb}{0,0,0}
\definecolor{Cerulean}{rgb}{0,0,0}
\definecolor{Fuchsia}{rgb}{0,0,0}

% Keywords (español e inglés)
\def\keywordsEn{\vspace{.5em}
{\textbf{\textit{Key words ---}}\,\relax%
}}
\def\endkeywordsEn{\par}

\def\keywordsEs{\vspace{.5em}
{\textbf{\textit{Palabras clave ---}}\,\relax%
}}
\def\endkeywordsEs{\par}


% Abstract (español e inglés)
\def\abstractEs{\vspace{.5em}
{\textbf{\textit{Resumen ---}}\,\relax%
}}
\def\endabstractEs{\par}

\def\abstractEn{\vspace{.5em}
{\textbf{\textit{Abstract ---}}\,\relax%
}}
\def\endabstractEn{\par}

% Estilo páginas de capítulos
\fancypagestyle{plain}{
\fancyhf{}
\fancyfoot[CO]{\footnotesize\emph{\nombretrabajo}}
\fancyfoot[RO]{\thepage}
\renewcommand{\footrulewidth}{.6pt}
\renewcommand{\headrulewidth}{0.0pt}
}

% Estilo resto de páginas
\pagestyle{fancy}

% Estilo páginas impares
\fancyfoot[CO]{\footnotesize\emph{\nombretrabajo}}
\fancyfoot[RO]{\thepage}
\rhead[]{\leftmark}

% Estilo páginas pares
\fancyfoot[CE]{\emph{\pieparcen}}
\fancyfoot[LE]{\thepage}
\fancyfoot[RE]{\pieparizq}
\lhead[\leftmark]{}

% Guía del pie de página
\renewcommand{\footrulewidth}{.6pt}

% Nombre de los bloques de código
\renewcommand{\lstlistingname}{Código}

% Estilo de los lstlistings
\lstset{
    frame=tb,
    breaklines=true,
    postbreak=\raisebox{0ex}[0ex][0ex]{\ensuremath{\color{gray}\hookrightarrow\space}}
}

% Definiciones de funciones para los títulos
\newlength\salto
\setlength{\salto}{3.5ex plus 1ex minus .2ex}
\newlength\resalto
\setlength{\resalto}{2.3ex plus.2ex}

% Estilo de los acrónimos
\renewcommand{\acronymname}{Glosario}
\renewcommand{\glossaryname}{Glosario}
\pretolerance=2000
\tolerance=3000

% Texto índice de tablas
\addto\captionsspanish{
\def\tablename{Tabla}
\def\listtablename{\'Indice de tablas}
}

% Traducir appendix/appendices
\renewcommand\appendixtocname{Apéndices}
\renewcommand\appendixpagename{Apéndices}

% Comando code (lstlisting sin cambio de página)
\lstnewenvironment{code}[1][]%
  { \noindent\minipage{0.935\linewidth}\medskip
    \vspace{5mm}
    \lstset{basicstyle=\ttfamily\footnotesize,#1}}
  {\endminipage}


% Definiciones de comandos
\newcommand{\figpdf}[3]{ % {FIGNAME}{FACTOR}{CAPTION} }
    \begin{figure}[tb]
        \centerline{\includegraphics[width=#2\textwidth]{graphics/#1}}
        \caption{#3}
        \label{fig:#1}
    \end{figure}
}

\newcommand{\figpdftwo}[3]{ % {FIGNAME}{FACTOR}{CAPTION} }
    \begin{figure}[h]
        \centerline{\includegraphics[width=#2\textwidth]{graphics/#1}}
        \caption{#3}
        \label{fig:#1}
    \end{figure}
}

\newcommand{\figpdftree}[3]{ % {FIGNAME}{FACTOR}{CAPTION} }
    \begin{figure}[h!]
        \centerline{\includegraphics[width=#2\textwidth]{graphics/#1}}
        \caption{#3}
        \label{fig:#1}
    \end{figure}
}
\newcommand{\nombreautor}{Author}
\newcommand{\nombreprimertutor}{First Tutor}
\newcommand{\nombresegundotutor}{Second Tutor}
\newcommand{\nombretrabajo}{Title}
\newcommand{\fecha}{Date}
\newcommand{\grado}{Master name}
% Descomentar si tu trabajo tiene un ponente
%\newcommand{\nombreponente}{TODO: Nombre del ponente}
% Descomentar si tu trabajo está asociado a un grupo de investigación
\newcommand{\grupoInvestigacion}{Research group}
\newcommand{\departamento}{Department}
\newcommand{\facultad}{Escuela Politécnica Superior}
\newcommand{\universidad}{Universidad Autónoma de Madrid}
\newcommand{\pieparizq}{}
\newcommand{\pieparcen}{Trabajo de Fin de Master}
\newcommand{\logoizq}{Logo_EPS}
\newcommand{\logoder}{Logo_UAM}
\newcommand{\correo}{e-mail}



% Glosario y acrónimos
%\makeglossaries
%% Acrónimos

% TODO: Añadir aquí los acrónimos
% Ejemplo de acrónimo
\newacronym{FPGA}{FPGA}{Field-Programmable Gate Array}

% Glosario

% TODO: Añadir aquí las definiciones del glosario
% Ejemplo de glosario
\newglossaryentry{bitstream}{name={bitstream},description={En este contexto se refiere al binario que configura el Hardware de la FPGA}}

% Rerefencias
\bibliography{src/bibliografia}

% Inicio del documento
\begin{document}

% Elección del idioma (español)
\selectlanguage{spanish}

%
% Portada
%
\pagenumbering{gobble}
%
% Portada
%

% Universidad, Facultad
\pagenumbering{Alph}

\begin{titlepage}
\selectlanguage{spanish}
\begin{center}
\textbf{\begin{huge}
\universidad \\
\end{huge}}
\bigskip 
\begin{LARGE}
\facultad \\
\end{LARGE}
\end{center}

\bigskip
\bigskip

%
% Imágenes (logos) izquierdo y derecho
%
\begin{figure}[h]
	\begin{center}
		\includegraphics[scale=0.35]{\logoizq}
    \hspace{1cm}
		\includegraphics[scale=0.4]{\logoder}
	\end{center}	
\end{figure}
\bigskip
\bigskip
\bigskip

\textbf{\begin{center}
\begin{large}
\MakeUppercase{Trabajo Fin de Master}
\end{large}
\end{center}}

\bigskip
\bigskip
\bigskip

% Nombre del TFM
\begin{center}
\textbf{\begin{huge}
\MakeUppercase{\nombretrabajo}\\
\end{huge}}
\end{center}

\bigskip
\bigskip
\bigskip

% Grado
\begin{center}
\begin{large}
\textbf{\grado}\\
\end{large}
\end{center}





% Nombre del autor
\vspace{\fill}
\begin{center}
\textbf{\nombreautor}\\
% Tutor
\textbf{Tutora: \nombreprimertutor}\\
% Tutor
\textbf{Tutor: \nombresegundotutor}\\
% Ponente, si está definido en main.tex
\ifcsname nombreponente\endcsname
\textbf{Ponente: \nombreponente}\\
\fi
\textbf{\departamento}
\bigskip

% Fecha
\textbf{\fecha}\\
\end{center}
\end{titlepage}

% Primera página

\thispagestyle{empty}
\par\vspace*{\fill}
\begin{flushleft}
\begin{scriptsize}
\end{scriptsize}\end{flushleft}
\newpage
\thispagestyle{empty}
\begin{center}

% Nombre del trabajo
\textbf{\begin{large}
\MakeUppercase{\nombretrabajo}\\*
\end{large}}
\vspace*{0.2cm}
\vspace{5cm}

% Nombre del autor y del tutor
\large Autor: \nombreautor \\*
\large Tutora: \nombreprimertutor \\*
\large Tutor: \nombresegundotutor \\*
\ifcsname nombreponente\endcsname
\large Ponente: \nombreponente\\
\fi

\vfill

% Grupo de investigación, departamento, facultad, universidad y fecha
\ifcsname grupoInvestigacion\endcsname
\grupoInvestigacion \\
\fi
\departamento \\
\facultad \\
\universidad \\
\vspace{1cm}
\fecha \\

\clearpage

\end{center}

\normalsize

\hypersetup{pageanchor=true}

% Estilo de párrafo de los capítulos
\setlength{\parskip}{0.75em}
\renewcommand{\baselinestretch}{1.25}
% Interlineado simple
\spacing{1}

%
% Agradecimientos
%
\pagenumbering{Roman}
\setcounter{page}{0}
\chapter*{Agradecimientos}

TODO: Agradecimientos.

Lorem ipsum dolor sit amet, consectetur adipiscing elit. Phasellus laoreet dolor at sodales porta. Morbi facilisis hendrerit lacus vel sollicitudin. Aenean eleifend urna metus, eget vestibulum libero dictum tincidunt. Curabitur quis ultrices lorem. Duis ultricies, eros eget condimentum pharetra, tellus eros lobortis nulla, vel mattis nibh dui et felis. Interdum et malesuada fames ac ante ipsum primis in faucibus. Nam non lorem et ligula condimentum molestie. Fusce quis dolor non metus suscipit commodo. Praesent vel pulvinar lectus. Nullam ac dui eget magna accumsan volutpat. Aliquam sed purus quis lorem dictum rutrum auctor eu enim. Pellentesque a urna ac ligula cursus lacinia. Aenean sodales justo massa, vel imperdiet justo imperdiet ut. Nulla euismod pulvinar arcu eu convallis. Vivamus a tempus nunc, et vulputate nulla.

Sed dapibus aliquam imperdiet. Vivamus est quam, fermentum vitae augue id, ultricies tincidunt massa. Praesent tincidunt ex sem, ut aliquet nulla imperdiet eu. Duis ac ultricies lorem. Aenean consequat ipsum nec arcu aliquam, sit amet interdum quam tempus. In justo odio, bibendum vel nulla nec, aliquet tristique justo. In vel metus ut libero suscipit ultricies.

Class aptent taciti sociosqu ad litora torquent per conubia nostra, per inceptos himenaeos. Proin urna elit, iaculis id quam at, pretium laoreet ipsum. Phasellus ultricies faucibus ex et eleifend. Quisque facilisis erat dolor, ac rhoncus erat convallis et. Aliquam semper eleifend imperdiet. Sed eros ipsum, sagittis in pellentesque vel, vestibulum a augue. Duis sapien mauris, fringilla a tortor ut, sollicitudin volutpat nunc. Pellentesque vestibulum vel arcu in molestie. Nullam fermentum dolor luctus metus efficitur pulvinar. Pellentesque risus enim, tempus id ullamcorper in, maximus id nisl. Cras rhoncus consequat augue eu gravida. Ut efficitur mauris vitae orci dignissim sagittis. Suspendisse vitae massa eget nunc bibendum interdum.


% Cita
\begin{flushright}
\textit{``TODO: Cita relevante''}
TODO: Autor de la cita
\end{flushright}
  

%
% Resumen
%
% Resumen en inglés

\chapter*{Abstract}

\begin{abstractEn}
Abstract


\end{abstractEn}

% Palabras clave en inglés
\begin{keywordsEn}
keywords
\end{keywordsEn}

% Resumen en español
\chapter*{Resumen}

\begin{abstractEs}
Resumen

\end{abstractEs}

% Palabras clave en español
\begin{keywordsEs}
Palabras clave
\end{keywordsEs}


%
% Glosario
%
\printglossary[title=Glosario,toctitle=Glosario]
\printglossary[title=Acrónimos,toctitle=Acrónimos,type=\acronymtype]

% Estilo de párrafo de los índices
\setlength{\parskip}{1pt}
\renewcommand{\baselinestretch}{1}

%
% Tabla de contenidos
%
\tableofcontents
\listoftables
\listoffigures
\cleardoublepage

% Estilo de párrafo de los capítulos
\setlength{\parskip}{0.75em}
\renewcommand{\baselinestretch}{1.25}
% Interlineado simple
\spacing{1}
% Numeración contenido
\pagenumbering{arabic}
\setcounter{page}{1}

%
% Introducción
%
\chapter{Introducción \label{sec:introduccion}}

\section{Motivaci\'on \label{sec:motivacion}}

 
\section{Estructura del documento \label{sec:estructura}}

%
% Estado del arte
%
\chapter{Estado del arte\label{sec:estado_del_arte}}


%
% Enfoque
%
\chapter{Enfoque\label{sec:enfoque}}


%
% arquitectura y herramienta
%
\chapter{Arquitectura y herramienta \label{sec:arquitectura_herra}}




%
% Resultados
%
\chapter{Evaluaci\'on\label{sec:evaluacion}}


%
% Conclusiones
%
\chapter{Conclusiones y trabajo futuro\label{sec:conclusiones_trabfu}}
\section{Conclusiones \label{sec:conclusiones}}

\section{Trabajo futuro \label{sec:trab_futuro}}

 

%
% Página en blanco
%
\cleardoublepage

%
% Bibliografía
%
\printbibliography[heading=bibintoc]

% No expandir elementos para llenar toda la página
\raggedbottom

%
% Apéndices
%
\appendix
\cleardoublepage
\addappheadtotoc
\appendixpage

%
% TODO: Apéndices del TFG
%
\chapter{Apendice A \label{apendiceA}}


% Fin del documento
\end{document}
